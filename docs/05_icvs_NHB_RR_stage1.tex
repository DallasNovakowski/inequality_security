% Options for packages loaded elsewhere
\PassOptionsToPackage{unicode}{hyperref}
\PassOptionsToPackage{hyphens}{url}
%
\documentclass[
  english,
  man]{apa6}
\usepackage{amsmath,amssymb}
\usepackage{lmodern}
\usepackage{ifxetex,ifluatex}
\ifnum 0\ifxetex 1\fi\ifluatex 1\fi=0 % if pdftex
  \usepackage[T1]{fontenc}
  \usepackage[utf8]{inputenc}
  \usepackage{textcomp} % provide euro and other symbols
\else % if luatex or xetex
  \usepackage{unicode-math}
  \defaultfontfeatures{Scale=MatchLowercase}
  \defaultfontfeatures[\rmfamily]{Ligatures=TeX,Scale=1}
\fi
% Use upquote if available, for straight quotes in verbatim environments
\IfFileExists{upquote.sty}{\usepackage{upquote}}{}
\IfFileExists{microtype.sty}{% use microtype if available
  \usepackage[]{microtype}
  \UseMicrotypeSet[protrusion]{basicmath} % disable protrusion for tt fonts
}{}
\makeatletter
\@ifundefined{KOMAClassName}{% if non-KOMA class
  \IfFileExists{parskip.sty}{%
    \usepackage{parskip}
  }{% else
    \setlength{\parindent}{0pt}
    \setlength{\parskip}{6pt plus 2pt minus 1pt}}
}{% if KOMA class
  \KOMAoptions{parskip=half}}
\makeatother
\usepackage{xcolor}
\IfFileExists{xurl.sty}{\usepackage{xurl}}{} % add URL line breaks if available
\IfFileExists{bookmark.sty}{\usepackage{bookmark}}{\usepackage{hyperref}}
\hypersetup{
  pdftitle={The Relationship between Inequality and Consumption of Security Products},
  pdfauthor={Dallas Novakowski1 \& Mehdi Mourali1},
  pdflang={en-EN},
  pdfkeywords={keywords},
  hidelinks,
  pdfcreator={LaTeX via pandoc}}
\urlstyle{same} % disable monospaced font for URLs
\usepackage{graphicx}
\makeatletter
\def\maxwidth{\ifdim\Gin@nat@width>\linewidth\linewidth\else\Gin@nat@width\fi}
\def\maxheight{\ifdim\Gin@nat@height>\textheight\textheight\else\Gin@nat@height\fi}
\makeatother
% Scale images if necessary, so that they will not overflow the page
% margins by default, and it is still possible to overwrite the defaults
% using explicit options in \includegraphics[width, height, ...]{}
\setkeys{Gin}{width=\maxwidth,height=\maxheight,keepaspectratio}
% Set default figure placement to htbp
\makeatletter
\def\fps@figure{htbp}
\makeatother
\setlength{\emergencystretch}{3em} % prevent overfull lines
\providecommand{\tightlist}{%
  \setlength{\itemsep}{0pt}\setlength{\parskip}{0pt}}
\setcounter{secnumdepth}{-\maxdimen} % remove section numbering
% Make \paragraph and \subparagraph free-standing
\ifx\paragraph\undefined\else
  \let\oldparagraph\paragraph
  \renewcommand{\paragraph}[1]{\oldparagraph{#1}\mbox{}}
\fi
\ifx\subparagraph\undefined\else
  \let\oldsubparagraph\subparagraph
  \renewcommand{\subparagraph}[1]{\oldsubparagraph{#1}\mbox{}}
\fi
% Manuscript styling
\usepackage{upgreek}
\captionsetup{font=singlespacing,justification=justified}

% Table formatting
\usepackage{longtable}
\usepackage{lscape}
% \usepackage[counterclockwise]{rotating}   % Landscape page setup for large tables
\usepackage{multirow}		% Table styling
\usepackage{tabularx}		% Control Column width
\usepackage[flushleft]{threeparttable}	% Allows for three part tables with a specified notes section
\usepackage{threeparttablex}            % Lets threeparttable work with longtable

% Create new environments so endfloat can handle them
% \newenvironment{ltable}
%   {\begin{landscape}\centering\begin{threeparttable}}
%   {\end{threeparttable}\end{landscape}}
\newenvironment{lltable}{\begin{landscape}\centering\begin{ThreePartTable}}{\end{ThreePartTable}\end{landscape}}

% Enables adjusting longtable caption width to table width
% Solution found at http://golatex.de/longtable-mit-caption-so-breit-wie-die-tabelle-t15767.html
\makeatletter
\newcommand\LastLTentrywidth{1em}
\newlength\longtablewidth
\setlength{\longtablewidth}{1in}
\newcommand{\getlongtablewidth}{\begingroup \ifcsname LT@\roman{LT@tables}\endcsname \global\longtablewidth=0pt \renewcommand{\LT@entry}[2]{\global\advance\longtablewidth by ##2\relax\gdef\LastLTentrywidth{##2}}\@nameuse{LT@\roman{LT@tables}} \fi \endgroup}

% \setlength{\parindent}{0.5in}
% \setlength{\parskip}{0pt plus 0pt minus 0pt}

% \usepackage{etoolbox}
\makeatletter
\patchcmd{\HyOrg@maketitle}
  {\section{\normalfont\normalsize\abstractname}}
  {\section*{\normalfont\normalsize\abstractname}}
  {}{\typeout{Failed to patch abstract.}}
\patchcmd{\HyOrg@maketitle}
  {\section{\protect\normalfont{\@title}}}
  {\section*{\protect\normalfont{\@title}}}
  {}{\typeout{Failed to patch title.}}
\makeatother
\shorttitle{Inequality and Security Consumption}
\keywords{keywords\newline\indent Word count: X}
\DeclareDelayedFloatFlavor{ThreePartTable}{table}
\DeclareDelayedFloatFlavor{lltable}{table}
\DeclareDelayedFloatFlavor*{longtable}{table}
\makeatletter
\renewcommand{\efloat@iwrite}[1]{\immediate\expandafter\protected@write\csname efloat@post#1\endcsname{}}
\makeatother
\usepackage{lineno}

\linenumbers
\usepackage{csquotes}
\ifxetex
  % Load polyglossia as late as possible: uses bidi with RTL langages (e.g. Hebrew, Arabic)
  \usepackage{polyglossia}
  \setmainlanguage[]{english}
\else
  \usepackage[main=english]{babel}
% get rid of language-specific shorthands (see #6817):
\let\LanguageShortHands\languageshorthands
\def\languageshorthands#1{}
\fi
\ifluatex
  \usepackage{selnolig}  % disable illegal ligatures
\fi
\newlength{\cslhangindent}
\setlength{\cslhangindent}{1.5em}
\newlength{\csllabelwidth}
\setlength{\csllabelwidth}{3em}
\newenvironment{CSLReferences}[2] % #1 hanging-ident, #2 entry spacing
 {% don't indent paragraphs
  \setlength{\parindent}{0pt}
  % turn on hanging indent if param 1 is 1
  \ifodd #1 \everypar{\setlength{\hangindent}{\cslhangindent}}\ignorespaces\fi
  % set entry spacing
  \ifnum #2 > 0
  \setlength{\parskip}{#2\baselineskip}
  \fi
 }%
 {}
\usepackage{calc}
\newcommand{\CSLBlock}[1]{#1\hfill\break}
\newcommand{\CSLLeftMargin}[1]{\parbox[t]{\csllabelwidth}{#1}}
\newcommand{\CSLRightInline}[1]{\parbox[t]{\linewidth - \csllabelwidth}{#1}\break}
\newcommand{\CSLIndent}[1]{\hspace{\cslhangindent}#1}

\title{The Relationship between Inequality and Consumption of Security Products}
\author{Dallas Novakowski\textsuperscript{1} \& Mehdi Mourali\textsuperscript{1}}
\date{}


\authornote{

The authors made the following contributions. Dallas Novakowski: Conceptualization, Writing - Original Draft Preparation, Writing - Review \& Editing, Analysis; Mehdi Mourali: Writing - Review \& Editing.

Correspondence concerning this article should be addressed to Dallas Novakowski, Postal address. E-mail: \href{mailto:dallas.novakowski1@ucalgary.ca}{\nolinkurl{dallas.novakowski1@ucalgary.ca}}

}

\affiliation{\vspace{0.5cm}\textsuperscript{1} University of Calgary, Haskayne School of Business}

\abstract{
A growing body of evidence suggests that economic inequality causes humans to take more risks and engage in aggressive behaviours (Payne, et al., 2017). In a ``winner-takes-all'' environment, risky activities such as gambling, lying, and crime can be a person's only means of accessing contested resources and goals.

There is comparitively little research investigating whether people anticipate risk-taking and conflict from their neighbors in unequal environments. An informative context for examining the fear-provoking effect of inequality is the security market, which offers goods that protect customers from the actions of malicious agents. For instance, barred windows are purchased because they (purportedly) protect consumers from break-ins. Given the costly nature of these security products, security consumption can only be worthwhile if a consumer expects that other people have harmful intentions.

The proposed research will examine whether inequality will increase consumers' willingness to purchase security products through two studies: 1) manipulations of inequality and distributional fairness in an economic game context, and 2) multilevel analysis of the International Crime Victimization Survey, a cross-national survey, totaling n = 52,909 respondents' experiences with crime, policing, crime prevention, and feelings of unsafety. Together, this project seeks to test the fear-provoking mechanism of inequality, and assess whether a country's income inequality is associated with individual-level consumption of security goods.
}



\begin{document}
\maketitle

\hypertarget{introduction}{%
\section{Introduction}\label{introduction}}

A growing body of evidence suggests that economic inequality leads to increased risk-taking and conflict in humans. In a ``winner-takes-all'' environment, individuals are more likely to resort to risky and criminal activities out of desperation, to effectively compete for contested positions and resources (Payne, Brown-Iannuzzi, \& Hannay, 2017). Despite modern societies becoming increasingly unequal (Saez \& Zucman, 2016), there is little research examining whether inequality causes people to \emph{expect} desperation and risk-taking from others.

Security products and services are one of the few goods that confer little intrinsic value to consumers. Whereas vacations are inherently satisfying, the costs of burglar alarms, sturdy locks, and security cameras are only worthwhile \emph{if consumers expect that other agents will cause them harm}. Despite the decreasing crime rates across the world (United Nations, 2017), demand for security is growing (projected to increase 29\% by 2025; (Markets and Markets, 2021). The choice to purchase security goods thus represents a substantial, and poorly understood, allocation of resources.

Criminologists have found that perceived victimization risk leads to greater security behaviours (Kanan \& Pruitt, 2002), while psychologists have documented the effects of inequality on crime and antisociality (Daly, 2016). However, there has been little integration of these streams of research. This project offers a contribution by testing a hypothesis that intersects the disciplines of criminology, psychology, and marketing: that consumers' perceived victimization risk, and subsequent consumption of security products, will grow with increasing economic inequality.

\hypertarget{social-interaction-as-a-security-problem}{%
\subsection{Social Interaction as a Security Problem}\label{social-interaction-as-a-security-problem}}

At its most comprehensive, security consumption could be considered as any prevention-focused activity, conducted by an agent seeking to minimize the expected costs from harms. Security consumption can be defined as an expenditure of resources in the consumer context that helps the agent to (a) recover from, (b) cope with, (c) avoid, and/or (d) deter hazards. However, this proposal will focus on the threats posed by other people, and consequently, the consumption of deterrence-focused goods. This emphasis on social sources of harm reveals how decision-makers must consider others' intentions when deciding whether to purchase security goods. Specifically, security behaviours are heavily influenced by the hypothetical imperative: \emph{if I expect others to \underline{try to steal}, my best response is to \underline{purchase security products}}; (Hargreaves-Heap \& Varoufakis, 2004).

Life-altering hazards can come from a variety of sources (natural disasters, economic instability, pandemics). However, other humans (via murder) are only outranked by mosquitoes as the deadliest animal to modern Homo sapiens (Gates, 2014). Likewise, social life carries the risk of property crimes and defection in collective action problems (e.g., commitments to monogamy, business contracts). Humans' sociality carries enormous benefits to individuals, including life-saving aid in times of need, high levels of skill specialization, and access to skills and knowledge gained by others (Henrich \& Gil-White, 2001). Given the substantial benefits and risks of social interaction, when will a person spend their limited time, money, and opportunities protecting themselves from other people?

When faced with the problem of consuming security goods to protect against social threats, it is critical for decision-makers to understand the beliefs, preferences, and intentions of their social partners. Mistaking a partner's willingness to inflict harms can result in wasted resources in the presence of a benevolent or indifferent partner, or can lead to a vulnerability being exploited by a hostile agent. Similar to organism's abilities to detect and respond to predators and pathogens, humans are responsive to environmental cues when deciding whether a social interaction is likely to be harmful (e.g., reciprocity, reputation, group membership; (Alves, Koch, \& Unkelbach, 2018; Schmid, Chatterjee, Hilbe, \& Nowak, 2021)

Purchasing security products against social threat is a potent way of measuring distrust amongst individuals, ``incur{[}ing{]} costs to mitigate their vulnerability.'' (McEvily, Radzevick, \& Weber, 2012, p. 287)

\hypertarget{inequality-and-risk-taking}{%
\subsection{Inequality and Risk-Taking}\label{inequality-and-risk-taking}}

Economic inequality likewise appears to be an important indicator of potentially harmful environments. Inequality has been linked to greater individual risk-taking (Mishra, Hing, \& Lalumière, 2015; Payne, Brown-Iannuzzi, \& Hannay, 2017), higher nation-level homicide rates (Daly, 2016), and greater interpersonal conflict and spiteful behaviours (Krupp \& Cook, 2018; Wobker, 2015).

Two main explanations have been proposed for how inequality at the population level influences individual-level risk-taking (De Courson \& Nettle, 2021). Firstly, compositional effects reflect how inequality increases risk-taking through absolute deprivation; societies with a skewed distribution of wealth tend to have higher poverty rates, and those in poverty are more likely to resort to criminal activities to have a chance at meeting their subsistence needs (Pridemore, 2008). Secondly, psychosocial effects describe how risk-taking is facilitated through social comparisons. In the presence of inequality, individuals who are well-off in absolute terms may be dissatisfied with their current endowments compared to others. High comparison standards can lead to more ambitious goals, and experiences of relative deprivation (i.e, envy, inequality aversion). As the psychological distance between one's current state and goal state becomes larger, individuals become more likely to resort to risky behaviours, in an effort to bridge the gap between themselves and their aspiration levels.

\hypertarget{the-anticipation-of-conflict---rationality-in-the-face-of-desparation---desperate-neighbors-are-a-security-hazard}{%
\subsection{\texorpdfstring{The anticipation of conflict - Rationality in the face of desparation - \textbf{Desperate neighbors are a security hazard}}{The anticipation of conflict - Rationality in the face of desparation - Desperate neighbors are a security hazard}}\label{the-anticipation-of-conflict---rationality-in-the-face-of-desparation---desperate-neighbors-are-a-security-hazard}}

Given the evidence for inequality's effect on individual risk-taking, people may use cues of economic disparities to infer that risky and hostile behaviours are more likely to occur. For instance, in the two-player ultimatum game, proposers most often offer 50\% of an endowment to their partner. About half of responders reject proposals less than 20\% of the total resource, causing both parties to get nothing (reviewed in Camerer, 2011).

Beyond altruism and the proposer's taste for fairness, generosity in the ultimatum game is in the proposer's approximate self-interest. If the responder is likely to reject profitable, but unequal offers, then a proposer's best chance to get any money is to offer an equal split of the funds. By failing to anticipate inequality aversion in their partner, a payoff-maximizing actor would offer the smallest nonzero sum possible, and would often have their offer rejected and obtain \emph{worse} outcomes.

Given that interpersonal security measures are specifically designed to deter antagonistic social partners, decision-makers must consider others' intentions when deciding whether to purchase security goods. Specifically, this project will examine whether the presence of economic inequality heightens participants' expectations of antagonism from their partners, and subsequently participants' willingness to engage in security consumption.

Thus, prospective consumers of security would do well to recognize the potentially corrosive effects of inequality on partners' intentions.

the proposition follows that \emph{1) when inequality increases (whether in income or wealth})\emph{, individuals will spend more on security goods}.

\hypertarget{protecting-from-inequality-driven-conflict}{%
\subsection{Protecting from inequality-driven conflict}\label{protecting-from-inequality-driven-conflict}}

\textbf{\emph{Push and pull hypothesis of inequality:}} Following unequal distribution of resources, disadvantaged individuals resort to risky behaviours to reduce the observed disparity, leading others (regardless of resource access) to seek protection against these risky and possibly antisocial actions.

attempt to maintain their positions.

\hypertarget{methods}{%
\section{Methods}\label{methods}}

\hypertarget{multilevel-analysis-of-the-international-crime-victimization-survey}{%
\subsection{Multilevel Analysis of the International Crime Victimization Survey}\label{multilevel-analysis-of-the-international-crime-victimization-survey}}

\textbf{The Methods section in Letters and Articles should ideally not exceed 3,000 words but may be longer if necessary}

The relationship between country-level inequality and the consumption of security products will be tested using a multilevel linear regression. This analysis will be accomplished using a combination of three archival datasets.

Firstly, indicators of security consumption have been accessed from the International Crime Victimization survey (ICVS, Van Kesteren, 2010), which is an accumulation of standardized sample surveys to look at householders' experiences with crime, policing, crime prevention and feelings of unsafety. Although it does not consist of longitudinal observations, the ICVS has been distributed across five phases over fifteen years (1989, 1992, 1996, 2000, 2005), surveying over 300,000 people across 78 different countries . For the purposes of this study, it is most notable that this survey contains items such as respondents' adopted measures to protect themselves against burglary.

Secondly, nation-level inequality in disposable income was accessed through the Standardized World Income Inequality Database (SWIID, Solt, 2020). The SWIID has been designed to maximize the comparability of income inequality data while maintaining the widest possible coverage across countries and over time. As a result, the gini coefficients are accompanied by standard errors to reflect uncertainty in the estimates.

Lastly, countries' expenditure-side real GDP was retrieved from the Penn World Table, version 10.0 (PWT). The selected GDP values are adjusted for Purchasing Power Parity, ``to compare relative living standards across countries at a single point in time'' (Feenstra, Inklaar, \& Timmer, 2015). Nations' GDP values were divided by their population sizes to yield a per-capita GDP value, which will be used for these analysis.

\hypertarget{proposed-sample-characteristics}{%
\subsubsection{Proposed sample characteristics}\label{proposed-sample-characteristics}}

Although the above datasets extend across multiple years, the current analysis will be limited to observations across 2004-2006. Imposing this restriction yields a sufficient initial number of cases 94,749 and countries 34, combined with relative recency of data.

\hypertarget{inclusionexclusion-criteria-e.g.-outliers}{%
\subsubsection{Inclusion/exclusion criteria (e.g., outliers)}\label{inclusionexclusion-criteria-e.g.-outliers}}

Since the ICVS, SWIID, and PWT are publicly available, preprocessing and inclusion/exclusion steps have already been implemented, and are available in the sourced and attached code. Some notable preprocessing steps are as follows:

\begin{enumerate}
\def\labelenumi{\arabic{enumi}.}
\item
  merged ICVS responses from United Kingdom, England \& Wales, Northern Ireland, and Scotland into a single cluster of ``United Kingdom,'' as SWIID and PWT values are not available for these portions of the UK.
\item
  Excluded participants who refused response on any measures
\item
  Excluded four countries that had missing responses on key predictor and outcome measures: Switzerland, Hong Kong, Estonia, and Japan
\item
  Averaged gini and gdp values across 2004-2006 into a single index
\end{enumerate}

After a conservative exclusion process for missing data, survey sweeps conducted across 2004-6 yielded 31 countries and a total 83,684 participants, with a minimum of 784 participants per country. An exclusion strategy maximizing the number of variables yields 28 countries, and 52,908 participants.

\hypertarget{measurement-and-variables}{%
\subsubsection{Measurement and variables}\label{measurement-and-variables}}

Security consumption have been operationalized as a count variable, summing respondents' self-reported ownership of five different preventative measures: a burglar alarm, special door locks, special grills, a high perimeter fence, and caretaker security. This count variable excludes several prevention variables due to missing data, such as owning a watch dog, having surveillance arrangements with neighbors, and purchasing insurance against criminal activities.

\hypertarget{predictor-variables}{%
\paragraph{Predictor variables}\label{predictor-variables}}

Few model

Inequality GDP

age, in five-year bins gender, binary employed income quartile

partnered

many model

Victimization experiences over the past five years has likewise been treated as a count variable of seven variables: car theft, theft from motor vehicle, theft of bicycle, burglary, attempted burglary, robbery, and personal theft.

peru is missing assaults

asutralia missing sexual offences

probably drop assaults and sexual offences

\hypertarget{experimental-procedures}{%
\subsubsection{experimental procedures}\label{experimental-procedures}}

\hypertarget{proposed-analysis-pipeline-including-all-preprocessing-steps-precise-description-of-all-planned-analyses}{%
\subsubsection{Proposed analysis pipeline, including all preprocessing steps, precise description of all planned analyses}\label{proposed-analysis-pipeline-including-all-preprocessing-steps-precise-description-of-all-planned-analyses}}

Analyses on archival data provides many degrees of freedom, not just in the processing of data, but in model construction as well.

\begin{enumerate}
\def\labelenumi{\arabic{enumi}.}
\tightlist
\item
  run many clusters
\item
  run few clusters - few variables
\end{enumerate}

iv\_2005

xtmixed total\_security num\_victim\_5yr gini\_cent gdppc\_2004\_6\_scale age\_cent employed male {[}pw=individual\_weight{]} \textbar\textbar{} country:, pwscale(size)

\begin{enumerate}
\def\labelenumi{\arabic{enumi}.}
\setcounter{enumi}{2}
\tightlist
\item
  few clusters - many variables
\item
  few clusters - many variables - winzorized \textbf{Report this one}
\item
  few clusters - many variables - flagged
\item
  many clusters - winzorized
\end{enumerate}

xtmixed security\_winz num\_victim\_5yr\_winz gini\_wc gdppc\_2004\_6\_ws age\_cent employed male {[}pw=individual\_weight{]} \textbar\textbar{} country:, pwscale(size)

\begin{enumerate}
\def\labelenumi{\arabic{enumi}.}
\setcounter{enumi}{6}
\tightlist
\item
  many clusters - flagged
\item
  Many clusters - with national victimization
\item
  few clusters- with
\end{enumerate}

\hypertarget{power-analysis-neymanpearson-inference}{%
\subsubsection{Power analysis (Neyman=Pearson inference)}\label{power-analysis-neymanpearson-inference}}

\textbf{For frequentist analysis plans, the a priori power must be 0.95 or higher for all proposed hypothesis tests.}

In the case of highly uncertain effect sizes, a variable sample size and interim data analysis is permissible but with inspection points stated in advance,

\hypertarget{full-descriptions-must-be-provided-of-any-outcome-neutral-criteria-that-must-be-met-for-successful-testing-of-the-stated-hypotheses.}{%
\subsubsection{Full descriptions must be provided of any outcome-neutral criteria that must be met for successful testing of the stated hypotheses.}\label{full-descriptions-must-be-provided-of-any-outcome-neutral-criteria-that-must-be-met-for-successful-testing-of-the-stated-hypotheses.}}

Such quality checks might include the absence of floor or ceiling effects in data distributions, positive controls, or other quality checks that are orthogonal to the experimental hypotheses.

\hypertarget{analysis-procedure}{%
\subsubsection{Analysis procedure}\label{analysis-procedure}}

\hypertarget{data-analysis}{%
\subsubsection{Data analysis}\label{data-analysis}}

\hypertarget{experiments-on-economic-inequality-in-the-security-game}{%
\subsection{Experiments on economic inequality in the Security Game}\label{experiments-on-economic-inequality-in-the-security-game}}

\hypertarget{the-security-game}{%
\paragraph{The security game}\label{the-security-game}}

Economic games examining self-protection are not new. Most notably, McEvily et al.~(2012) devised a behavioural distrust game, whereby participants choose whether to incur a cost that guarantees that their partner splits a surplus income equally. Compared to the distrust game, the security game is explicitly probabilistic - partners have risky choice to take funds for themselves, and security expenditures reduce the probability of successful ``attacks.''

\hypertarget{proposed-sample-characteristics-1}{%
\subsubsection{Proposed sample characteristics}\label{proposed-sample-characteristics-1}}

\hypertarget{inclusionexclusion-criteria-e.g.-outliers-1}{%
\subsubsection{Inclusion/exclusion criteria (e.g., outliers)}\label{inclusionexclusion-criteria-e.g.-outliers-1}}

\hypertarget{measurement-and-variables-1}{%
\subsubsection{Measurement and variables}\label{measurement-and-variables-1}}

\hypertarget{experimental-procedures-1}{%
\subsubsection{experimental procedures}\label{experimental-procedures-1}}

\hypertarget{proposed-analysis-pipeline-including-all-preprocessing-steps-precise-description-of-all-planned-analyses-1}{%
\subsubsection{Proposed analysis pipeline, including all preprocessing steps, precise description of all planned analyses}\label{proposed-analysis-pipeline-including-all-preprocessing-steps-precise-description-of-all-planned-analyses-1}}

\hypertarget{power-analysis-neymanpearson-inference-1}{%
\subsubsection{Power analysis (Neyman=Pearson inference)}\label{power-analysis-neymanpearson-inference-1}}

\hypertarget{analysis-procedure-1}{%
\subsubsection{Analysis procedure}\label{analysis-procedure-1}}

\hypertarget{data-analysis-1}{%
\subsubsection{Data analysis}\label{data-analysis-1}}

\hypertarget{timeline-for-completion-of-the-study-and-proposed-resubmission-date-if-stage-1-review-is-successful}{%
\subsection{Timeline for completion of the study and proposed resubmission date if Stage 1 review is successful}\label{timeline-for-completion-of-the-study-and-proposed-resubmission-date-if-stage-1-review-is-successful}}

We used R {[}Version 4.1.1; (\textbf{R-base?}){]} and the R-package \emph{papaja} {[}Version 0.1.0.9997; (\textbf{R-papaja?}){]} for all our analyses.

\hypertarget{discussion}{%
\section{Discussion}\label{discussion}}

\newpage

\hypertarget{references}{%
\section{References}\label{references}}

\begingroup
\setlength{\parindent}{-0.5in}
\setlength{\leftskip}{0.5in}

\hypertarget{refs}{}
\begin{CSLReferences}{1}{0}
\leavevmode\hypertarget{ref-alves2018}{}%
Alves, H., Koch, A., \& Unkelbach, C. (2018). A Cognitive-Ecological Explanation of Intergroup Biases. \emph{Psychological Science}, \emph{29}(7), 1126--1133. \url{https://doi.org/10.1177/0956797618756862}

\leavevmode\hypertarget{ref-camerer2011}{}%
Camerer, C. F. (2011). \emph{Behavioral Game Theory: Experiments in Strategic Interaction}. Princeton University Press.

\leavevmode\hypertarget{ref-daly2016}{}%
Daly, M. (2016). \emph{Killing the competition} (1 edition). New Brunswick: Transaction Publishers.

\leavevmode\hypertarget{ref-decourson2021}{}%
De Courson, B., \& Nettle, D. (2021). Why do inequality and deprivation produce high crime and low trust? \emph{Scientific Reports}, \emph{11}(1), 1937. \url{https://doi.org/10.1038/s41598-020-80897-8}

\leavevmode\hypertarget{ref-feenstra2015}{}%
Feenstra, R. C., Inklaar, R., \& Timmer, M. P. (2015). The Next Generation of the Penn World Table. \emph{American Economic Review}, \emph{105}(10), 3150--3182. \url{https://doi.org/10.1257/aer.20130954}

\leavevmode\hypertarget{ref-gates2014}{}%
Gates, B. (2014). The deadliest animal in the world. Retrieved from \url{https://www.gatesnotes.com/Health/Most-Lethal-Animal-Mosquito-Week}

\leavevmode\hypertarget{ref-hargreaves-heap2004}{}%
Hargreaves-Heap, S., \& Varoufakis, Y. (2004). \emph{Game theory: A critical introduction} (2nd ed.). London: Routledge. \url{https://doi.org/10.4324/9780203489291}

\leavevmode\hypertarget{ref-henrich2001}{}%
Henrich, J., \& Gil-White, F. J. (2001). The evolution of prestige: freely conferred deference as a mechanism for enhancing the benefits of cultural transmission. \emph{Evolution and Human Behavior}, \emph{22}(3), 165--196. \url{https://doi.org/10.1016/S1090-5138(00)00071-4}

\leavevmode\hypertarget{ref-kanan2002}{}%
Kanan, J. W., \& Pruitt, M. V. (2002). Modeling Fear of Crime and Perceived Victimization Risk: The (In)Significance of Neighborhood Integration. \emph{Sociological Inquiry}, \emph{72}(4), 527--548. \url{https://doi.org/10.1111/1475-682X.00033}

\leavevmode\hypertarget{ref-krupp2018}{}%
Krupp, D. B., \& Cook, T. R. (2018). Local Competition Amplifies the Corrosive Effects of Inequality. \emph{Psychological Science}, \emph{29}(5), 824--833. \url{https://doi.org/10.1177/0956797617748419}

\leavevmode\hypertarget{ref-marketsandmarkets2021}{}%
Markets and Markets. (2021). Physical security market size, share and global market forecast to 2025 \textbar{} COVID-19 impact analysis \textbar{} MarketsandMarkets. Retrieved from \url{https://www.marketsandmarkets.com/Market-Reports/physical-security-market-1014.html}

\leavevmode\hypertarget{ref-mcevily2012}{}%
McEvily, B., Radzevick, J. R., \& Weber, R. A. (2012). Whom do you distrust and how much does it cost? An experiment on the measurement of trust. \emph{Games and Economic Behavior}, \emph{74}(1), 285--298. \url{https://doi.org/10.1016/j.geb.2011.06.011}

\leavevmode\hypertarget{ref-mishra2015}{}%
Mishra, S., Hing, L. S. S., \& Lalumière, M. L. (2015). Inequality and Risk-Taking. \emph{Evolutionary Psychology}, \emph{13}(3), 147470491559629. \url{https://doi.org/10.1177/1474704915596295}

\leavevmode\hypertarget{ref-payne2017}{}%
Payne, B. K., Brown-Iannuzzi, J. L., \& Hannay, J. W. (2017). Economic inequality increases risk taking. \emph{Proceedings of the National Academy of Sciences}, \emph{114}(18), 4643--4648. \url{https://doi.org/10.1073/pnas.1616453114}

\leavevmode\hypertarget{ref-pridemore2008}{}%
Pridemore, W. A. (2008). A Methodological Addition to the Cross-National Empirical Literature on Social Structure and Homicide: A First Test of the Poverty-Homicide Thesis*. \emph{Criminology}, \emph{46}(1), 133--154. \url{https://doi.org/10.1111/j.1745-9125.2008.00106.x}

\leavevmode\hypertarget{ref-saez2016}{}%
Saez, E., \& Zucman, G. (2016). Wealth inequality in the united states since 1913: Evidence from capitalized income tax data *. \emph{The Quarterly Journal of Economics}, \emph{131}(2), 519--578. \url{https://doi.org/10.1093/qje/qjw004}

\leavevmode\hypertarget{ref-schmid2021}{}%
Schmid, L., Chatterjee, K., Hilbe, C., \& Nowak, M. A. (2021). A unified framework of direct and indirect reciprocity. \emph{Nature Human Behaviour}. \url{https://doi.org/10.1038/s41562-021-01114-8}

\leavevmode\hypertarget{ref-solt2020}{}%
Solt, F. (2020). Measuring Income Inequality Across Countries and Over Time: The Standardized World Income Inequality Database. \emph{Social Science Quarterly}, \emph{101}(3), 1183--1199. \url{https://doi.org/10.1111/ssqu.12795}

\leavevmode\hypertarget{ref-unitednations2017}{}%
United Nations. (2017). Reports on world crime trends. Retrieved from \href{https:////www.unodc.org/unodc/en/data-and-analysis/statistics/reports-on-world-crime-trends.html}{//www.unodc.org/unodc/en/data-and-analysis/statistics/reports-on-world-crime-trends.html}

\leavevmode\hypertarget{ref-vankesteren2010}{}%
Van Kesteren, J. N. (2010). European Survey on Crime and Safety - EU ICS 2005International Crime Victims Surveys - ICVS - 1989, 1992, 1996, 2000, 2005. \url{https://doi.org/10.17026/DANS-XNJ-RMB2}

\leavevmode\hypertarget{ref-wobker2015}{}%
Wobker, I. (2015). The Price of Envy{{}}An Experimental Investigation of Spiteful Behavior. \emph{Managerial and Decision Economics}, \emph{36}(5), 326--335. \url{https://doi.org/10.1002/mde.2672}

\end{CSLReferences}

\endgroup


\end{document}
